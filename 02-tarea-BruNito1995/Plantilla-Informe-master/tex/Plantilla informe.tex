\documentclass[a4paper, 11pt, spanish]{article}

% ---------------- Paquetes de formato.
\usepackage[spanish]{babel} % Para codificar el texto.
\usepackage[top=70mm, bottom=30mm, left=18mm, right=18mm]{geometry} % Para modificar el tamaño de las hojas.
\usepackage{fancyhdr} % Para poner la institución, dpto y curso arriba.
\usepackage{parallel} % Para escribir en columnas (Poner los integrantes a la derecha).
\usepackage[firstpage=false]{background} % Para poner el logo de la U arriba en todas las páginas.
\usepackage{enumerate} % Para poder enumerar como a), b), etc.
\usepackage[utf8]{inputenc} % Para usar acentos en vez de \'.

% ---------------- Paquetes graficos.
\usepackage{graphicx} % remove the demo option.
\usepackage{tikz}
\usepackage{caption}
\usepackage{subcaption} % Para usar sub-figuras

% ---------------- Paquetes matematicos.
\usepackage{amsmath} % Para poder hacer N^o y se vea bonito.
\usepackage{amsthm} % Fonts matematicos.
\usepackage{amssymb} % Para usar \therefore
\usepackage{commath} % Para usar \abs y \norm

% ---------------- Paquetes 'computines'.
\usepackage{listings} % Para escribir codigo y que se vea bonito.
\usepackage[% Descomentar las opciones a usar.
spanish,
%boxed, % Encierra los algoritmos en un cuadro.
boxruled, % Encierra los algoritmos en un cuadro colocando el titulo al comienzo.
%ruled, % Coloca una linea al comienzo y otra al final del algoritmo. El titulo de este queda al comienzo del algoritmo.
%algoruled, % Lo mismo que el anterior pero mas espaciado.
%tworuled, % Como ruled pero sin una linea al comienzo.
%algochapter, % Los algoritmos se enumeran segun capitulo.
%algopart, % Los algoritmos se enumeran por partes.
%figure, % Los algoritmos son considerados figuras (y por ende salen en \listoffigures).
%linesnumbered, % Enumera las lineas.
longend % Los end son para cada ciclo, por ejemplo endif para los if-else.
]{algorithm2e}

% ---------------- Paquetes miscelaneos.
%\usepackage{lipsum} % Para hacer placeholders.
\usepackage{bohr} % Para dibujar atomos.

% ---------------- Comando mas corto para insertar figuras. Ojo que deben estar guardadas en ./img/
% \fig{name}{width}{height}{caption}
\newcommand{\fig}[4]{%
	\begin{figure}[!htbp]
		\centering
		\includegraphics[width=#2, height=#3]{img/#1}
		\caption{#4}
	\end{figure}
}
% ejemplo:
%\fig{nombre_imagen.png}{10cm}{5cm}{Titulo de la imagen}

%%% Ejemplo para poner dos imagenes juntas, lado a lado

%\begin{figure}[!ht]
%\centering
%\begin{subfigure}{.5\textwidth}
%  \centering
%  \includegraphics[width=10cm, height=8cm]{img/curva0f1.pdf}
%  \caption{Curva de nivel 0 de $F_{1}$.}
%  \label{fig:sub1}
%\end{subfigure}%
%\begin{subfigure}{.5\textwidth}
%  \centering
%  \includegraphics[width=10cm, height=8cm]{img/curva0f2.pdf}
%  \caption{Curva de nivel 0 de $F_{2}$.}
%  \label{fig:sub2}
%\end{subfigure}
%\caption{Curvas de nivel 0 de $F_{1}$ y $F_{2}$}
%\label{fig:test}
%\end{figure}


% ---------------- Comando para hacer itemes
% \Solution{pregunta}{solucion}
\newcommand{\Solution}[2]{%
	\item #1 \vspace{0.2cm}
	\textbf{Soluci\'on:} #2
}
% \Demonstration{pregunta}{demostracion}
\newcommand{\Demonstration}[2]{%
	\item #1 \vspace{0.2cm}
	\begin{proof}
		#2	
	\end{proof}
}

% ---------------- Opciones de algortihm2e
\SetKw{KwRequire}{Require:}

% ---------------- Opciones de background.
\SetBgColor{black}
\SetBgScale{1}
\SetBgOpacity{1}
\SetBgAngle{0}
\SetBgContents{%
	\begin{tikzpicture}[remember picture,overlay]
		\node at (-8.0,0.746\textheight) {\includegraphics[height=18mm,width= 0.155\textwidth]{img/LogoUIngenieria.png}};
	\end{tikzpicture}
}

% ---------------- Creacion de institucion, departamento y curso.
\fancyheadoffset[L]{-2cm}
\fancyhead[L]{\footnotesize{\textbf{\textsf{Universidad de Chile \\ Facultad de Cs. F\'isicas y Matem\'aticas \\ Departamento de F\'isica \\ Métodos Númericos : FI3104-1}}}}
\renewcommand{\headrulewidth}{0pt}
\setlength{\voffset}{-3cm}

\pagestyle{fancy} % Estilo de las páginas

\begin{document}

\pagenumbering{gobble} % Quita el numero de las paginas (y las resetea a 1)

\clearpage

\thispagestyle{fancy}
\vspace*{6.5cm} % Espacio vertical para posicionar bien el título (En una de esas esto se puede optimizar para que no sea tan a la fuerza bruta).

% ---------------- Titulo.
\begin{center}
	\Large{\textbf{\textsf{Informe Tarea 2}}} \\
	\huge{\textbf{\textsf{Algoritmo de Búsqueda de Ceros}}}
\end{center}

\vspace*{7.5cm}

% ---------------- Integrantes, profes, etc.
\begin{Parallel}{1cm}{7.5cm}
	\ParallelRText{%
		\begin{flushright} % Tira el texto hacia la derecha.
			\large{%
				\textsf{%
					\begin{tabular}{rl}
						Alumno: &
							\begin{tabular}[t]{@{}l@{}}
								Bruno Quezada
							\end{tabular} \\
						Profesore: & 
							\begin{tabular}[t]{@{}l@{}}
						 		Valentino González
							\end{tabular} \\
						Auxiliares: &
							\begin{tabular}[t]{@{}l@{}}
								José Vines. \\
								Jou-Hui Ho.
							\end{tabular} \\	 
					\end{tabular} \\
					Fecha: \today
				}
			}
		\end{flushright}
	}
\end{Parallel}

\clearpage

\pagenumbering{arabic} % Numeros de pagina Arabicos (y los resetea a 1)

\newpage

\tableofcontents % Indice. Descomente para usar.
%\listoffigures % Lista de figuras. Descomente para usar.
%\listoftables % Lista de tablas. Descomente para usar.

\newpage
\section{Pregunta 1}
\subsection{Introducci\'on}
\noindent Para esta pregunta se solicita encontrar el largo de un cable entre 2 torres separadas por $20[m]$, con una caída de $7.5[m]$ en su punto medio. La ecuación que modela el la forma que adopta el cable es la catenaria definida como $Cat(x,x_0,\alpha) = \frac{\alpha}{2}(e^{\frac{x-x_0}{\alpha}} + e^{-\frac{x-x_0}{\alpha}})$. Para la resolución de este problema es necesario primero encontrar el $\alpha$ que cumpla las condiciones establecidas para el cable, esto se hará por medio de la definición de una función auxiliar, la cual se anula en un punto de interés, y se utilizará un algoritmo que busque las raíces de una función. Una vez encontrado $\alpha$ queda determinada la forma del cable y a través de la integración de $\int_0^{20} \sqrt{Cat'(x,x_0,\alpha)^2 + 1} dx $ se puede determinar su largo.

\subsection{Procedimiento}
Primero es necesario definir una función auxiliar adecuada que al anularse nos permita conocer el valor de $\alpha$ para este caso particular. Se utiliza $f(x,x_0,\alpha) = cat(x,x_0,\alpha) + 7.5$ de modo que al evaluar en $x = 10$ y $x_0 = 10$ se anule cuando el cable alcance los $7.5 [m]$ de caída. En primera instancia se intentó utilizar el método de newton para encontrar el cero por su rápida convergencia, pero no es posible, ya que al derivar la catenaria con respecto a $\alpha$ se obtiene: $\dfrac{d (cat)}{d \alpha} = \dfrac{ x-x_0}{ 4 \alpha }(e^{ -2 \dfrac{x-x_0}{\alpha}   } - e^{ 2 \dfrac{x-x_0}{\alpha}   })$ y al evaluar en $x = 10$ y $x_0 = 10$ la derivada se anula. Por lo tanto, se implementa el método de la bisección para resolver el problema.\\
\noindent Con un poco de álgebra es trivial reconocer que $\alpha = -7.5$ cumple las condiciones del problema así que el intervalo para la bisección se define entre $[-5,-8]$.\\
\noindent Una vez encontrado $\alpha$ se procede a calcular el largo del cable, esto se logra a través de

\subsection{Resultados}

\subsection{Conclusiones y Discusi\'on}




\pagebreak
\section{Pregunta 2}

\subsection{Introducci\'on}

\subsection{Procedimiento}

\subsection{Resultados}

\subsection{Conclusiones y Discusi\'on}

\end{document}
