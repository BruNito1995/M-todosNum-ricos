\documentclass[letter, 11pt]{article}
%% ================================
%% Packages =======================
\usepackage[utf8]{inputenc}      %%
\usepackage[T1]{fontenc}         %%
\usepackage{lmodern}             %%
\usepackage[spanish]{babel}      %%
\decimalpoint                    %%
\usepackage{fullpage}            %%
\usepackage{fancyhdr}            %%
\usepackage{graphicx}            %%
\usepackage{amsmath}             %%
\usepackage{color}               %%
\usepackage{mdframed}            %%
\usepackage[colorlinks]{hyperref}%%
%% ================================
%% ================================

%% ================================
%% Page size/borders config =======
\setlength{\oddsidemargin}{0in}  %%
\setlength{\evensidemargin}{0in} %%
\setlength{\marginparwidth}{0in} %%
\setlength{\marginparsep}{0in}   %%
\setlength{\voffset}{-0.5in}     %%
\setlength{\hoffset}{0in}        %%
\setlength{\topmargin}{0in}      %%
\setlength{\headheight}{54pt}    %%
\setlength{\headsep}{1em}        %%
\setlength{\textheight}{8.5in}   %%
\setlength{\footskip}{0.5in}     %%
%% ================================
%% ================================

%% =============================================================
%% Headers setup, environments, colors, etc.
%%
%% Header ------------------------------------------------------
\fancypagestyle{firstpage}
{
  \fancyhf{}
  \lhead{\includegraphics[height=4.5em]{LogoDFI.jpg}}
  \rhead{FI3104-1 \semestre\\
         Métodos Numéricos para la Ciencia e Ingeniería\\
         Prof.: \profesor}
  \fancyfoot[C]{\thepage}
}

\pagestyle{plain}
\fancyhf{}
\fancyfoot[C]{\thepage}
%% -------------------------------------------------------------
%% Environments -------------------------------------------------
\newmdenv[
  linecolor=gray,
  fontcolor=gray,
  linewidth=0.2em,
  topline=false,
  bottomline=false,
  rightline=false,
  skipabove=\topsep
  skipbelow=\topsep,
]{ayuda}
%% -------------------------------------------------------------
%% Colors ------------------------------------------------------
\definecolor{gray}{rgb}{0.5, 0.5, 0.5}
%% -------------------------------------------------------------
%% Aliases ------------------------------------------------------
\newcommand{\scipy}{\texttt{scipy}}
%% -------------------------------------------------------------
%% =============================================================
%% =============================================================================
%% CONFIGURACION DEL DOCUMENTO =================================================
%% Llenar con la información pertinente al curso y la tarea
%%
\newcommand{\tareanro}{10}
\newcommand{\fechaentrega}{6/12/2018 23:59 hrs}
\newcommand{\semestre}{2018B}
\newcommand{\profesor}{Valentino González}
%% =============================================================================
%% =============================================================================


\begin{document}
\thispagestyle{firstpage}

\begin{center}
  {\uppercase{\LARGE \bf Tarea \tareanro}}\\
  Fecha de entrega: \fechaentrega
\end{center}


%% =============================================================================
%% ENUNCIADO ===================================================================

\noindent{\large \bf Problema 1}

El archivo adjunto \texttt{correlacion.dat} contiene 3 columnas:


\noindent - \emph{year}: el año de la medición.

\noindent - \emph{N\_MPhD}: el número total de Ph.D.'s en Matemáticas otorgados
  durante el correspondiente año en Estados Unidos.

\noindent - \emph{U}: la reserva de Uranio enriquecido (peso en libras) almacenado en
  plantas nucleares de Estados Unidos durante el correspondiente año (no
  pregunten cómo lo averigué, pero los datos son reales).

\vspace{1em}
Queremos responder 2 preguntas sutilmente distintas:

\noindent\textbf{1)} ¿Cuánto deben bajar las reservas de Uranio para que nadie
obtenga un doctorado en Matemáticas?

\noindent\textbf{2)} ¿Cuántos Ph.D. se deberían graduar en un año para que las
reservas de Uranio bajaran a cero?

Para responder estas preguntas, necesitamos extrapolar la correlación que
existe entre estas dos cantidades.

\vspace{0.5em}
\noindent\textbf{Responda las siguientes preguntas:}
\renewcommand{\labelenumi}{(\alph{enumi})}
\begin{enumerate}

\item El archivo contiene 13 filas, es decir, 13 puntos. ¿Por qué es mala idea
  fitear un polinomio de orden 12 a los puntos para obtener respuesta a
  nuestras preguntas 1) y 2)?

\item Responda la pregunta 1) ajustando un polinomio $p = p(U)$ de orden 1 a
  los 13 puntos y encontrando dónde es que el polinomio intersecta al eje
  \emph{y} (\emph{=N\_MPhD}). (Puede que la respuesta sea una cantidad de
  Uranio negativa, es decir, imposible que no haya doctorados en matemáticas).
  Para hacer este ajuste defina una función de mérito como $\chi^2$ y busque su
  mínimo. Tiene libertad de usar el método que quiera pero describa en detalle
  qué hizo.

\item Responda la pregunta 2) ajustando un polinomio $p = p(N\_MPhD)$ de orden
  1 a los 13 puntos y encontrando donde es que el polinomio intersecta al eje
  \emph{y} ($=U$). Mismo método que en la parte anterior pero ajustando $U$
  como función de $N\_MPhD$ en vez de $N\_MPhD$ como función de $U$.

\item Ahora dese cuenta de que los polinomios de las partes b y c \underline{no
  son equivalentes}. Intente responder la pregunta 1 usando el polinomio de la
  parte c y la pregunta 2 usando el polinomio de la parte b y verá que no son
  equivalentes (también puede graficar los dos polinomios juntos). ¿Por qué es
  que no son equivalentes?

  (Si quiere, puede intentar escribir una función de mérito tipo $\chi^2$ pero
  que sí sea simétrica, es decir, que el resultado sea independiente de si
  ajustamos $N\_MPhD$ como función de $U$ o si ajustamos $U$ como función de
  $N\_MPhD$.)

\item Explore qué pasa cuando usa polinomios de orden mayor para responder las
  mismas preguntas y explique sus resultados. Explore al menos un polinomio de
  orden 5 y estudie su comportamiento.

\end{enumerate}

\noindent\textbf{*IMPORTANTE}: Obviamente nada de esto tiene sentido,
correlación no implica causalidad.


\pagebreak
\noindent{\large \bf Problema 2}

\begin{ayuda}
  En el problema anterior ajustamos un polinomio a un set de datos, un problema
  clásico de regresión lineal que puede resolverse de manera exacta invirtiendo
  una matriz. En el problema a continuación exploramos una función no-lineal.
  En este caso necesitamos otro algoritmo distinto para la minimización, por
  ejemplo, un algoritmo tipo Levenberg-Marquardt.
\end{ayuda}

La técnica de la espectroscopía, que se usa en múltiples disciplinas, consiste
en estudiar la radiación emitida por una fuente como función de la longitud de
onda. Las características de los espectros observados tales como la intensidad
y forma del contínuo y de las líneas de emisión y absorción, nos permiten
entender las propiedades físicas del ente emisor de la radiación.

En la figura a continuación, se observa un segmento del espectro de una fuente
que muesta radiación contínua (con una leve pendiente) y una línea de emisión.
Las unidades son de energía emitida por unidad de tiempo por unidad de área por
unidad de frecuencia $f_\nu [{\rm erg\ s^{-1} Hz^{-1} cm^{-2}}]$ vs.  longitud
de onda en [Angstrom]. Su trabajo consiste en modelar simultáneamente el
contínuo y la línea de absorción de la figura (los datos los encontrará en el
archivo \texttt{espectro.dat}).

\begin{center}
\includegraphics[width=0.6\textwidth]{spectrum.png}
\end{center}

Las líneas de emisión son en teoría casi infinitamente delgadas (hay un
ensanchamiento intrínseco dado por el principio de incertidumbre pero es muy
pequeño). Las observaciones, sin embargo, siempre muestran líneas mucho más
anchas. Dependiendo del mecanismo físico que produce el ensanchamiento, la
forma de la línea será distinta. Ud. deberá modelar la línea asumiendo el
mechanismo de ensanchamiento más típico, el cual produce líneas gaussianas.

El modelo completo será el de una línea recta para el contínuo (2 parámetros)
más una función gaussiana con 3 parámetros: amplitud, centro y varianza. Es
decir, debe modelar 5 parámetros a la vez.

  \begin{ayuda}
    \texttt{scipy.stats.norm} implementa la función Gaussiana si no quiere
    escribirla Ud. mismo. La forma de usarla es la siguiente: \texttt{g = A *
    scipy.stats.norm(loc=mu, scale=sigma).pdf(x)}; donde x es la longitud de
    onda donde evaluar la función.
  \end{ayuda}

Produzca un gráfico que muestre el espectro observado y el mejore fit obtenido.
Provea una tabla con los mejores parámetros, su error estándard y el valor de
$\chi^2_{\rm reducido}$ en su mínimo.


\begin{ayuda}
Para buscar el mejor modelo recuerde que es importante dar un punto de partida
que sea cercano para que los algoritmos converjan de manera efectiva. Ud. debe
idear un método para buscar ese punto de partida.
\end{ayuda}


\vspace{1em}
\noindent{\large \bf Problema 3}

En la Tarea nro. 3. utilizamos datos del \emph{Goddard Institute for Space
Science} (GISS) sobre la temperatura del planeta. Más específicamente,
utilizamos el archivo \texttt{GLB.Ts+dSST-short.csv} que contiene información
sobre las anomalías de temperatura medidas en la Tierra y los océanos (vea la
Tarea 3 y la página oficial
\href{https://data.giss.nasa.gov/gistemp/}{https://data.giss.nasa.gov/gistemp/}
para más detalles).

Utilizando estos mismos datos (versión más completa \texttt{GLB.Ts+dSST.csv}),
y asumiendo que no hacemos nada para alterar la tendencia mostrada por los
datos, estime para qué año la temperatura habrá cambiado en 2 grados celsius.

Explique en detalle cómo hizo la estimación (por ejemplo, ¿qué forma
paramétrica escogió?) y qué algoritmo de minimización usó. Provea un intervalo
de confianza para el año de la catástrofe más grande en la historia de la
humanidad
(\href{http://theconversation.com/why-is-climate-changes-2-degrees-celsius-of-warming-limit-so-important-82058}{artículo
random al respecto}).


\vspace{2em}
\noindent{\bf Instrucciones importantes.}
\begin{itemize}

\item {\bf Por 0.5 ptos. extra} utilizables en esta tarea. Hay una imágen
  escondida en el repositorio (en un \texttt{commit} antiguo). Revise la
  bitácora (\texttt{git log}) del repositorio para averiguar donde está.
  Recupérela sin usar github web (averigue cómo hacerlo, hay multiples
  tutoriales en la web) y agréguela al final de su informe. Describa los pasos
  que siguió para recuperar la imágen.  Recuerde que no puede simplemente
  buscarla en la interfaz web de github, la idea es que aprenda a recuperar
  archivos en un repositorio cualquiera de git (independiente de github.com).

\item Repartición de puntaje: 40\% implementación y resolución de los
   problemas (independiente de la calidad de su código); 45\% calidad del
   reporte entregado: demuestra comprensión de los problemas y su solución,
   claridad del lenguaje, calidad de las figuras utilizadas; 5\% aprueba a no
   \texttt{PEP8}; 10\% diseño del código: modularidad, uso efectivo de nombres
   de variables y funciones, docstrings, \underline{uso de git}, etc. Los 3
   problemas tienen igual peso en la nota final.

\item El informe debe ser entregado en formato \texttt{pdf}, este debe ser
  claro sin información de más ni de menos. \textbf{Esto es muy importante, no
  escriba de más, esto no mejorará su nota sino que al contrario}. La presente
  tarea probablemente no requiere informes de más de 5 páginas. Asegúrese de
  utilizar figuras efectivas y tablas para resumir sus resultados. Revise su
  ortografía.

  \item Evaluaremos su uso correcto de \texttt{python}. Si define una función
  relativametne larga o con muchos parámetros, recuerde escribir el
  \emph{docstring} que describa los parámetros que recibe la función, el
  output, y el detalle de qué es lo que hace la función. Recuerde que
  generalmente es mejor usar varias funciones cortas (que hagan una sola cosa
  bien) que una muy larga (que lo haga todo).  Utilice nombres explicativos
  tanto para las funciones como para las variables de su código. El mejor
  nombre es aquel que permite entender qué hace la función sin tener que leer
  su implementación ni su \emph{docstring}.

\item Su código debe aprobar la guía sintáctica de estilo
  (\href{https://www.python.org/dev/peps/pep-0008/}{\texttt{PEP8}}). En
  \href{http://pep8online.com}{esta página} puede chequear si su código aprueba
  \texttt{PEP8}.

\item Utilice \texttt{git} durante el desarrollo de la tarea para mantener un
  historial de los cambios realizados. La siguiente
  \href{https://education.github.com/git-cheat-sheet-education.pdf}{cheat
    sheet} le puede ser útil. {\bf Revisaremos el uso apropiado de la
  herramienta y asignaremos una fracción del puntaje a este ítem.} Realice
  cambios pequeños y guarde su progreso (a través de \emph{commits})
  regularmente. No guarde código que no corre o compila (si lo hace por algún
  motivo deje un mensaje claro que lo indique). Escriba mensajes claros que
  permitan hacerse una idea de lo que se agregó y/o cambió de un
  \texttt{commit} al siguiente.

\item Al hacer el informe usted debe decidir qué es interesante y agregar las
  figuras correspondientes. No olvide anotar los ejes, las unidades e incluir
  una \emph{caption} o título que describa el contenido de cada figura.

\item La tarea se entrega subiendo su trabajo a github. Trabaje en el código y
  en el informe, haga \textit{commits} regulares y cuando haya terminado
  asegúrese de hacer un último \texttt{commit} y luego un \texttt{push} para
  subir todo su trabajo a github. \textbf{REVISE SU REPOSITORIO PARA ASEGURARSE
  QUE SUBIÓ LA TAREA. SI UD. NO PUEDE VER SU INFORME EN GITHUB.COM, TAMPOCO
PODREMOS NOSOTROS.}

\end{itemize}
%% FIN ENUNCIADO ===============================================================
%% =============================================================================

\end{document}
