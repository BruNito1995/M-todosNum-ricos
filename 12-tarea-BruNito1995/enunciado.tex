\documentclass[letter, 11pt]{article}
%% ================================
%% Packages =======================
\usepackage[utf8]{inputenc}      %%
\usepackage[T1]{fontenc}         %%
\usepackage{lmodern}             %%
\usepackage[spanish]{babel}      %%
\usepackage{fullpage}            %%
\usepackage{fancyhdr}            %%
\usepackage{graphicx}            %%
\usepackage{amsmath}             %%
\usepackage{color}               %%
\usepackage{mdframed}            %%
\usepackage[colorlinks]{hyperref}%%
%% ================================
%% ================================

%% ================================
%% Page size/borders config =======
\setlength{\oddsidemargin}{0in}  %%
\setlength{\evensidemargin}{0in} %%
\setlength{\marginparwidth}{0in} %%
\setlength{\marginparsep}{0in}   %%
\setlength{\voffset}{-0.5in}     %%
\setlength{\hoffset}{0in}        %%
\setlength{\topmargin}{0in}      %%
\setlength{\headheight}{54pt}    %%
\setlength{\headsep}{1em}        %%
\setlength{\textheight}{8.5in}   %%
\setlength{\footskip}{0.5in}     %%
%% ================================
%% ================================

%% =============================================================
%% Headers setup, environments, colors, etc.
%%
%% Header ------------------------------------------------------
\fancypagestyle{firstpage}
{
  \fancyhf{}
  \lhead{\includegraphics[height=4.5em]{LogoDFI.jpg}}
  \rhead{FI3104-1 \semestre\\
         Métodos Numéricos para la Ciencia e Ingeniería\\
         Prof.: \profesor}
  \fancyfoot[C]{\thepage}
}

\pagestyle{plain}
\fancyhf{}
\fancyfoot[C]{\thepage}
%% -------------------------------------------------------------
%% Environments -------------------------------------------------
\newmdenv[
  linecolor=gray,
  fontcolor=gray,
  linewidth=0.2em,
  topline=false,
  bottomline=false,
  rightline=false,
  skipabove=\topsep
  skipbelow=\topsep,
]{ayuda}
%% -------------------------------------------------------------
%% Colors ------------------------------------------------------
\definecolor{gray}{rgb}{0.5, 0.5, 0.5}
%% -------------------------------------------------------------
%% Aliases ------------------------------------------------------
\newcommand{\scipy}{\texttt{scipy}}
%% -------------------------------------------------------------
%% =============================================================
%% =============================================================================
%% CONFIGURACION DEL DOCUMENTO =================================================
%% Llenar con la información pertinente al curso y la tarea
%%
\newcommand{\tareanro}{12}
\newcommand{\fechaentrega}{22/12/2018 23:59 hrs}
\newcommand{\semestre}{2018B}
\newcommand{\profesor}{Valentino González}
%% =============================================================================
%% =============================================================================


\begin{document}
\thispagestyle{firstpage}

\begin{center}
  {\uppercase{\LARGE \bf Tarea \tareanro}}\\
  Fecha de entrega: \fechaentrega
\end{center}


%% =============================================================================
%% ENUNCIADO ===================================================================
\noindent{\large \bf Problema 1}

En esta Tarea continuaremos nuestro análisis de los datos del \emph{Goddard
Institute for Space Science} (GISS) sobre la temperatura del planeta pero esta
vez utilizaremos un análisis Bayesiano de los datos.

Considere nuevamente los datos del archivo \texttt{GLB.Ts+dSST.csv}, en
específico, la columna \texttt{``J-D''} que indica la anomalía de temperatura
promedio en el período Enero--Diciembre de cada año. Probaremos los siguientes
dos modelos paramétricos:\\

\noindent Modelo 1: $JD = a_0 + a_1 \times yr + a_2 \times yr^2$; en este modelo
los parámetros son $(a_0, a_1, a_2)$.

\noindent Modelo 2: $JD = A_0 + exp((yr-yr_0) / \tau)$; parámetros $(A_0, yr_0,
\tau)$.\\

\noindent En ambos modelos $JD$ corresponde a la columna \texttt{``J-D''} y
$yr$ a la columna \texttt{``Year''}.

\begin{enumerate}

  \item Utilizando primero un estimador de máxima verosimilitud, encuentre los
    mejores parámetros para ambos modelos.

  \item Ahora utilice el teorema de Bayes para calcular la probabilidad
    posterior para los parámetros de cada uno de los modelos. Note que en el
    modelo 2, el dominio de $yr_0$ y $\tau$ son los reales positivos. Indique y
    justifique claramente las probabilidades a priori que utilizó. Reporte las
    probabilidades marginales para cada parámetro mediante un gráfico, la
    esperanza $E[\theta]$ para cada parámetro y el intervalo de credibilidad del
    68\% (el equivalente al intervalo de confianza).

  \item Finalmente utlice comparación Bayesiana de modelos para decidir qué
    modelo es el que repesenta mejor los datos. Justifique claramente su
    elección.

\end{enumerate}


\noindent{\bf Instrucciones importantes.}
\begin{itemize}

\item Repartición de puntaje: 40\% implementación y resolución del problema
  (independiente de la calidad de su código); 45\% calidad del reporte
  entregado: demuestra comprensión de los problemas y su solución, claridad del
  lenguaje, calidad de las figuras utilizadas; 5\% aprueba a no \texttt{PEP8};
  10\% diseño del código: modularidad, uso efectivo de nombres de variables y
  funciones, docstrings, \underline{uso de git}, etc.

\item El informe debe ser entregado en formato \texttt{pdf}, este debe ser
  claro sin información de más ni de menos. \textbf{Esto es muy importante, no
  escriba de más, esto no mejorará su nota sino que al contrario}. La presente
  tarea probablemente no requiere informes de más de 4 páginas. Asegúrese de
  utilizar figuras efectivas y tablas para resumir sus resultados. Revise su
  ortografía.

  \item Evaluaremos su uso correcto de \texttt{python}. Si define una función
  relativametne larga o con muchos parámetros, recuerde escribir el
  \emph{docstring} que describa los parámetros que recibe la función, el
  output, y el detalle de qué es lo que hace la función. Recuerde que
  generalmente es mejor usar varias funciones cortas (que hagan una sola cosa
  bien) que una muy larga (que lo haga todo).  Utilice nombres explicativos
  tanto para las funciones como para las variables de su código. El mejor
  nombre es aquel que permite entender qué hace la función sin tener que leer
  su implementación ni su \emph{docstring}.

\item Su código debe aprobar la guía sintáctica de estilo
  (\href{https://www.python.org/dev/peps/pep-0008/}{\texttt{PEP8}}). En
  \href{http://pep8online.com}{esta página} puede chequear si su código aprueba
  \texttt{PEP8}.

\item Utilice \texttt{git} durante el desarrollo de la tarea para mantener un
  historial de los cambios realizados. La siguiente
  \href{https://education.github.com/git-cheat-sheet-education.pdf}{cheat
    sheet} le puede ser útil. {\bf Revisaremos el uso apropiado de la
  herramienta y asignaremos una fracción del puntaje a este ítem.} Realice
  cambios pequeños y guarde su progreso (a través de \emph{commits})
  regularmente. No guarde código que no corre o compila (si lo hace por algún
  motivo deje un mensaje claro que lo indique). Escriba mensajes claros que
  permitan hacerse una idea de lo que se agregó y/o cambió de un
  \texttt{commit} al siguiente.

\item Al hacer el informe usted debe decidir qué es interesante y agregar las
  figuras correspondientes. No olvide anotar los ejes, las unidades e incluir
  una \emph{caption} o título que describa el contenido de cada figura.

\item La tarea se entrega subiendo su trabajo a github. Trabaje en el código y
  en el informe, haga \textit{commits} regulares y cuando haya terminado
  asegúrese de hacer un último \texttt{commit} y luego un \texttt{push} para
  subir todo su trabajo a github. \textbf{REVISE SU REPOSITORIO PARA ASEGURARSE
  QUE SUBIÓ LA TAREA. SI UD. NO PUEDE VER SU INFORME EN GITHUB.COM, TAMPOCO
PODREMOS NOSOTROS.}

\end{itemize}

%% FIN ENUNCIADO ===============================================================
%% =============================================================================

\end{document}
