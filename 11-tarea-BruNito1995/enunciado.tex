\documentclass[letter, 11pt]{article}
%% ================================
%% Packages =======================
\usepackage[utf8]{inputenc}      %%
\usepackage[T1]{fontenc}         %%
\usepackage{lmodern}             %%
\usepackage[spanish]{babel}      %%
\usepackage{fullpage}            %%
\usepackage{fancyhdr}            %%
\usepackage{graphicx}            %%
\usepackage{amsmath}             %%
\usepackage{color}               %%
\usepackage{mdframed}            %%
\usepackage[colorlinks]{hyperref}%%
%% ================================
%% ================================

%% ================================
%% Page size/borders config =======
\setlength{\oddsidemargin}{0in}  %%
\setlength{\evensidemargin}{0in} %%
\setlength{\marginparwidth}{0in} %%
\setlength{\marginparsep}{0in}   %%
\setlength{\voffset}{-0.5in}     %%
\setlength{\hoffset}{0in}        %%
\setlength{\topmargin}{0in}      %%
\setlength{\headheight}{54pt}    %%
\setlength{\headsep}{1em}        %%
\setlength{\textheight}{8.5in}   %%
\setlength{\footskip}{0.5in}     %%
%% ================================
%% ================================

%% =============================================================
%% Headers setup, environments, colors, etc.
%%
%% Header ------------------------------------------------------
\fancypagestyle{firstpage}
{
  \fancyhf{}
  \lhead{\includegraphics[height=4.5em]{LogoDFI.jpg}}
  \rhead{FI3104-1 \semestre\\
         Métodos Numéricos para la Ciencia e Ingeniería\\
         Prof.: \profesor}
  \fancyfoot[C]{\thepage}
}

\pagestyle{plain}
\fancyhf{}
\fancyfoot[C]{\thepage}
%% -------------------------------------------------------------
%% Environments -------------------------------------------------
\newmdenv[
  linecolor=gray,
  fontcolor=gray,
  linewidth=0.2em,
  topline=false,
  bottomline=false,
  rightline=false,
  skipabove=\topsep
  skipbelow=\topsep,
]{ayuda}
%% -------------------------------------------------------------
%% Colors ------------------------------------------------------
\definecolor{gray}{rgb}{0.5, 0.5, 0.5}
%% -------------------------------------------------------------
%% Aliases ------------------------------------------------------
\newcommand{\scipy}{\texttt{scipy}}
%% -------------------------------------------------------------
%% =============================================================
%% =============================================================================
%% CONFIGURACION DEL DOCUMENTO =================================================
%% Llenar con la información pertinente al curso y la tarea
%%
\newcommand{\tareanro}{11}
\newcommand{\fechaentrega}{15/12/2018 23:59 hrs}
\newcommand{\semestre}{2018B}
\newcommand{\profesor}{Valentino González}
%% =============================================================================
%% =============================================================================


\begin{document}
\thispagestyle{firstpage}

\begin{center}
  {\uppercase{\LARGE \bf Tarea \tareanro}}\\
  Fecha de entrega: \fechaentrega
\end{center}


%% =============================================================================
%% ENUNCIADO ===================================================================
\noindent{\large \bf Problema 1}

En la tarea anterior ajustamos un modelo gaussiano a un espectro que contiene
una línea de emisión (\texttt{espectro.dat}) y estimamos la incertidumbre
asociada a los mejores parámetros para el modelo pero sin información sobre la
incertidumbre asociada a los datos. Ahora expandiremos nuestro estudio
agregando: 1) un modelo lorentziano y 2) decidiremos si los modelos son o no
apropiados para modelar los datos.

Complete los siguientes pasos:

\renewcommand{\labelenumi}{(\arabic{enumi})}
\begin{enumerate}

\item Primero repita el modelo de la tarea anterior, es decir, modele la línea
  como si tuviese una forma Gaussiana. El modelo completo será el de una línea
  recta para el contínuo (2 parámetros) más una función gaussiana con 3
  parámetros: amplitud, centro y varianza. Es decir, debe modelar 5 parámetros
  a la vez.

  \begin{ayuda}
    \texttt{scipy.stats.norm} implementa la función Gaussiana si no quiere
    escribirla Ud. mismo. La forma de usarla es la siguiente: \texttt{g = A *
    scipy.stats.norm(loc=mu, scale=sigma).pdf(x)}; donde x es la longitud de
    onda donde evaluar la función.
  \end{ayuda}

\item El segundo modelo más típico de ensanchamiento corresponde al llamado
  perfil de Lorentz, Ud. debe implementarlo en este paso. De nuevo su modelo
  será el de una línea recta pero esta vez menos un perfil de Lorentz que tiene
  3 parámetros: amplitud, centro y varianza.  Nuevamente son 5 parámetros.

  \begin{ayuda}
    \texttt{scipy.stats.cauchy} implementa el perfil de Lorenz. Un ejemplo de
    su uso sería: \texttt{l = A * scipy.stats.cauchy(loc=mu,
    scale=sigma).pdf(x)}.
  \end{ayuda}

  Produzca un gráfico que muestre el espectro observado y los dos mejores fits
  obtenidos con cada uno de los modelos (gaussiano vs. lorentziano). Provea
  también una tabla con los mejores parámetros de cada modelo (con sus unidades
  correspondientes) y el valor de $\chi^2$ en su mínimo.

\item {\bf Idoneidad de los modelos.} En esta parte evaluaremos si cada uno de
  los modelos es buenos o no para representar los datos.

  \begin{ayuda}
    \small
    En este ejercicio en particular, ninguno de los dos modelos es el modelo
    exacto a partir del cual se creó el espectro.
  \end{ayuda}

  Debido a que los errores asociados a este problema son desconocidos, los
  tests de $\chi^2$ pueden no ser significativos para este caso por lo que
  utilizaremos un test de Kolmogorov-Smirnov para determinar la probabilidad
  asociada a la hipótesis nula de cada modelo (asuminedo que, al menos los
  errores desconocidos son aproximadamente gaussianos). Utilice el test para
  determinar: a) si los modelos son aceptables, y b) ¿puede determinar cuál
  modelo es mejor de acuerdo a este test?

\end{enumerate}


\pagebreak
\noindent{\bf Instrucciones importantes.}
\begin{itemize}

\item Repartición de puntaje: 40\% implementación y resolución del los
   problemas (independiente de la calidad de su código; 10\% parte 1, 10\%
   parte 2, 20\% parte 3); 45\% calidad del
   reporte entregado: demuestra comprensión de los problemas y su solución,
   claridad del lenguaje, calidad de las figuras utilizadas; 5\% aprueba a no
   \texttt{PEP8}; 10\% diseño del código: modularidad, uso efectivo de nombres
   de variables y funciones, docstrings, \underline{uso de git}, etc.

\item El informe debe ser entregado en formato \texttt{pdf}, este debe ser
  claro sin información de más ni de menos. \textbf{Esto es muy importante, no
  escriba de más, esto no mejorará su nota sino que al contrario}. La presente
  tarea probablemente no requiere informes de más de 4 páginas. Asegúrese de
  utilizar figuras efectivas y tablas para resumir sus resultados. Revise su
  ortografía.

  \item Evaluaremos su uso correcto de \texttt{python}. Si define una función
  relativametne larga o con muchos parámetros, recuerde escribir el
  \emph{docstring} que describa los parámetros que recibe la función, el
  output, y el detalle de qué es lo que hace la función. Recuerde que
  generalmente es mejor usar varias funciones cortas (que hagan una sola cosa
  bien) que una muy larga (que lo haga todo).  Utilice nombres explicativos
  tanto para las funciones como para las variables de su código. El mejor
  nombre es aquel que permite entender qué hace la función sin tener que leer
  su implementación ni su \emph{docstring}.

\item Su código debe aprobar la guía sintáctica de estilo
  (\href{https://www.python.org/dev/peps/pep-0008/}{\texttt{PEP8}}). En
  \href{http://pep8online.com}{esta página} puede chequear si su código aprueba
  \texttt{PEP8}.

\item Utilice \texttt{git} durante el desarrollo de la tarea para mantener un
  historial de los cambios realizados. La siguiente
  \href{https://education.github.com/git-cheat-sheet-education.pdf}{cheat
    sheet} le puede ser útil. {\bf Revisaremos el uso apropiado de la
  herramienta y asignaremos una fracción del puntaje a este ítem.} Realice
  cambios pequeños y guarde su progreso (a través de \emph{commits})
  regularmente. No guarde código que no corre o compila (si lo hace por algún
  motivo deje un mensaje claro que lo indique). Escriba mensajes claros que
  permitan hacerse una idea de lo que se agregó y/o cambió de un
  \texttt{commit} al siguiente.

\item Al hacer el informe usted debe decidir qué es interesante y agregar las
  figuras correspondientes. No olvide anotar los ejes, las unidades e incluir
  una \emph{caption} o título que describa el contenido de cada figura.

\item La tarea se entrega subiendo su trabajo a github. Trabaje en el código y
  en el informe, haga \textit{commits} regulares y cuando haya terminado
  asegúrese de hacer un último \texttt{commit} y luego un \texttt{push} para
  subir todo su trabajo a github. \textbf{REVISE SU REPOSITORIO PARA ASEGURARSE
  QUE SUBIÓ LA TAREA. SI UD. NO PUEDE VER SU INFORME EN GITHUB.COM, TAMPOCO
PODREMOS NOSOTROS.}

\end{itemize}

%% FIN ENUNCIADO ===============================================================
%% =============================================================================

\end{document}
