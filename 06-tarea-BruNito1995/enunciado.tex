\documentclass[letter, 11pt]{article}
%% ================================
%% Packages =======================
\usepackage[utf8]{inputenc}      %%
\usepackage[T1]{fontenc}         %%
\usepackage{lmodern}             %%
\usepackage[spanish]{babel}      %%
\decimalpoint                    %%
\usepackage{fullpage}            %%
\usepackage{fancyhdr}            %%
\usepackage{graphicx}            %%
\usepackage{amsmath}             %%
\usepackage{color}               %%
\usepackage{mdframed}            %%
\usepackage[colorlinks]{hyperref}%%
%% ================================
%% ================================

%% ================================
%% Page size/borders config =======
\setlength{\oddsidemargin}{0in}  %%
\setlength{\evensidemargin}{0in} %%
\setlength{\marginparwidth}{0in} %%
\setlength{\marginparsep}{0in}   %%
\setlength{\voffset}{-0.5in}     %%
\setlength{\hoffset}{0in}        %%
\setlength{\topmargin}{0in}      %%
\setlength{\headheight}{54pt}    %%
\setlength{\headsep}{1em}        %%
\setlength{\textheight}{8.5in}   %%
\setlength{\footskip}{0.5in}     %%
%% ================================
%% ================================

%% =============================================================
%% Headers setup, environments, colors, etc.
%%
%% Header ------------------------------------------------------
\fancypagestyle{firstpage}
{
  \fancyhf{}
  \lhead{\includegraphics[height=4.5em]{LogoDFI.jpg}}
  \rhead{FI3104-1 \semestre\\
         Métodos Numéricos para la Ciencia e Ingeniería\\
         Prof.: \profesor}
  \fancyfoot[C]{\thepage}
}

\pagestyle{plain}
\fancyhf{}
\fancyfoot[C]{\thepage}
%% -------------------------------------------------------------
%% Environments -------------------------------------------------
\newmdenv[
  linecolor=gray,
  fontcolor=gray,
  linewidth=0.2em,
  topline=false,
  bottomline=false,
  rightline=false,
  skipabove=\topsep
  skipbelow=\topsep,
]{ayuda}
%% -------------------------------------------------------------
%% Colors ------------------------------------------------------
\definecolor{gray}{rgb}{0.5, 0.5, 0.5}
%% -------------------------------------------------------------
%% Aliases ------------------------------------------------------
\newcommand{\scipy}{\texttt{scipy}}
%% -------------------------------------------------------------
%% =============================================================
%% =============================================================================
%% CONFIGURACION DEL DOCUMENTO =================================================
%% Llenar con la información pertinente al curso y la tarea
%%
\newcommand{\tareanro}{6}
\newcommand{\fechaentrega}{10/11/2018 23:59 hrs}
\newcommand{\semestre}{2018B}
\newcommand{\profesor}{Valentino González}
%% =============================================================================
%% =============================================================================


\begin{document}
\thispagestyle{firstpage}

\begin{center}
  {\uppercase{\LARGE \bf Tarea \tareanro}}\\
  Fecha de entrega: \fechaentrega
\end{center}


%% =============================================================================
%% ENUNCIADO ===================================================================
\noindent{\large \bf Problema 1}

Software Carpentry es una fundación sin fines de lucro que tiene como objetivo
enseñar a científicos e ingenieros de diversas ramas las habilidades necesarias
\emph{"to get more done in less time, and with less pain"}.

El siguiente tutorial desarrollado por Software Carpentry les ayudará a mejorar
el diseño del software que desarrollen para sus tareas y en el futuro. Estudien
el tutorial y respondan las preguntas que se encuentran a continuación.
Incluya sus respuestas en el informe. No se complique haciéndolas calzar con el
formato del informe, simplemente responda las preguntas.

El tutorial en la página de Software Carpentry
(\href{https://v4.software-carpentry.org/invperc/index.html}{link}) incluye las
diapositivas y una transcripción del video. También está disponible como
youtube playlist
(\href{https://www.youtube.com/playlist?list=PL5859017B018F03F4}{link}). Les
recomiendo reproducir el video a velocidad $1.5\times - 2\times$.

\noindent{\bf Preguntas:}
\begin{itemize}

  \item Describa la idea de escribir el \texttt{main driver} primero y llenar
    los huecos luego. ¿Por qué es buena idea?

\item ¿Cuál es la idea detrás de la función \texttt{mark\_filled}? ¿Por qué es
  buena idea crearla en vez del código original al que reemplaza?

\item ¿Qué es refactoring?

\item ¿Por qué es importante implmentar tests que sean sencillos de escribir?
  ¿Cuál es la estrategia usada en el tutorial?

\item El tutorial habla de dos grandes ideas para optimizar programas, ¿cuáles
  son esas ideas? Descríbalas.

\item ¿Qué es \emph{lazy evaluation}?

\item Describa la \emph{``other moral''} del tutorial (es una de las más
  importantes a la hora de escribir buen código).

\end{itemize}


\noindent{\large \bf Problema 2}

Para planetas que orbitan {\bf cerca del Sol}, el potencial gravitacional se
puede escribir como:

$$U(r) = - \dfrac{GMm}{r} + \alpha\dfrac{GMm}{r^2}$$

\noindent donde $\alpha$ es un número pequeño. Esta corrección a la ley de gravitación de
Newton es una buena aproximación derivada de la teoría de la relatividad
general de Einstein.

Bajo este potencial, las órbitas siguen siendo planas pero ya no son cerradas,
sino que precesan, es decir, el afelio (punto más alejado del Sol en la órbita)
gira alrededor del Sol.

\begin{enumerate}

  \item El archivo llamado \texttt{planeta.py} contiene el esqueleto de la
    clase \texttt{Planeta}. Ud. debe implementar los métodos de esa clase. Los
    docstrings explican en qué debe consistir cada método. Ud. tiene libertad
    de mejorar los docstrings, y agregar atributos y métodos a la clase según
    le parezca conveniente para resolver el problema descrito a continuación.

    El archivo llamado \texttt{solucion\_usando\_rk4.py} muestra cómo incluir
    la clase \texttt{Planeta} en un script separado. Ud. también puede resolver
    todo dentro del mismo archivo, en cuyo caso puede descartar
    \texttt{solucion\_usando\_rk4.py}.

  \item Parta por estudiar el caso $\alpha=0$ y considere las siguientes
    condiciones iniciales:
    \begin{flalign*}
      x_0 &= 10\\
      y_0 &= 0\\
      v_x &= 0\\
    \end{flalign*}

    Además, utilice unidades tales que $GMm = 1$ y escoja $v_y$ según le
    parezca (pero asegúrese de que la energía total sea negativa).

    Integre la ecuación de movimiento por aproximadamente 5 órbitas usando los
    métodos de RK4 y Verlet. Plotee las órbitas y la energía total del sistema
    como función del tiempo en los 2 casos. Comente los resultados.

  \item Ahora considere el caso $\alpha=10^{-2.XXX}$ (donde XXX son los 3
    últimos digitos de su RUT, antes del dígito verificador). Integre la
    ecuación de movimiento usando el método de Verlet por al menos 30 órbitas.
    Determine la velocidad angular de precesión. ¿Cómo lo hizo? En particular,
    ¿cómo determinó la posición del afelio? Grafique la órbita y la energía
    como función del tiempo.

\end{enumerate}

\begin{ayuda}
  \small
  \noindent {\bf Comentario.}
  Esta tarea pide explicitamente que utilice OOP (Object Oriented Programming)
  para su desarrollo. Es un ejercicio útil para aprender esta técnica.
\end{ayuda}

\vspace{1em}
\noindent\textbf{Instrucciones Importantes.}
\begin{itemize}

\item El algoritmo RK4 está implementado en muchas librerías y en la tarea
  pasada Ud. ya lo implementó. Si quiere, puede (recomendado) utilizar su
  propia implementación pero también puede utilizar otra de uso libre. El
  algoritmo de Verlet, sin embargo, lo debe implementar Ud. para esta tarea
  (puede usar cualquiera de sus variantes).

\item Evaluaremos su uso correcto de \texttt{python}. Si define una función
  relativametne larga o con muchos parámetros, recuerde escribir el
  \emph{docstring} que describa los parámetros que recibe la función, el
  output, y el detalle de qué es lo que hace la función. Recuerde que
  generalmente es mejor usar varias funciones cortas (que hagan una sola cosa
  bien) que una muy larga (que lo haga todo).  Utilice nombres explicativos
  tanto para las funciones como para las variables de su código. El mejor
  nombre es aquel que permite entender qué hace la función sin tener que leer
  su implementación ni su \emph{docstring}.

\item Su código debe aprobar la guía sintáctica de estilo
  (\href{https://www.python.org/dev/peps/pep-0008/}{\texttt{PEP8}}). En
  \href{http://pep8online.com}{esta página} puede chequear si su código aprueba
  \texttt{PEP8}.

\item Utilice \texttt{git} durante el desarrollo de la tarea para mantener un
  historial de los cambios realizados. La siguiente
  \href{https://education.github.com/git-cheat-sheet-education.pdf}{cheat
    sheet} le puede ser útil. {\bf Revisaremos el uso apropiado de la
  herramienta y asignaremos una fracción del puntaje a este ítem.} Realice
  cambios pequeños y guarde su progreso (a través de \emph{commits})
  regularmente. No guarde código que no corre o compila (si lo hace por algún
  motivo deje un mensaje claro que lo indique). Escriba mensajes claros que
  permitan hacerse una idea de lo que se agregó y/o cambió de un
  \texttt{commit} al siguiente.

\item Para hacer un informe completo Ud. debe decidir qué es interesante y
  agregar las figuras correspondientes. No olvide anotar los ejes e incluir una
  \emph{caption} o título que describa el contenido de cada figura. Tampoco
  olvide las unidades asociadas a las cantidades mostradas en los diferentes
  plots.

\item La tarea se entrega subiendo su trabajo a github. Clone este repositorio
  (el que está en su propia cuenta privada), trabaje en el código y en el
  informe y cuando haya terminado asegúrese de hacer un último \texttt{commit}
  y luego un \texttt{push} para subir todo su trabajo a github.

\item El informe debe ser entregado en formato \texttt{pdf}, este debe ser
  claro sin información de más ni de menos. \textbf{Esto es muy importante, no
  escriba de más, esto no mejorará su nota sino que al contrario}. La presente
  tarea probablemente no requiere informes de más de 4 o 5 páginas en total
  (dependiendo de cuántas figuras incluya; esto no es una regla estricta, sólo
  una referencia útil).  Asegúrese de utilizar figuras efectivas y tablas para
  resumir sus resultados. Revise su ortografía.

 \item No olvide indicar su RUT en el informe.

 \item Repartición de puntaje: P1: 30\%; P2: 70\%. El P2 además, se subdivide
   de la siguiente forma: 40\% implementación y resolución del problema
   (independiente de la calidad de su código); 45\% calidad del reporte
   entregado: demuestra comprensión del problema y su solución, claridad del
   lenguaje, calidad de las figuras utilizadas; 5\% aprueba o no \texttt{PEP8};
   10\% diseño del código: modularidad, uso efectivo de nombres de variables y
   funciones, docstrings, \underline{uso de git}, etc.

\end{itemize}

%% FIN ENUNCIADO ===============================================================
%% =============================================================================

\end{document}
